\documentclass[authoryear,preprint,review,12pts]{elsarticle}

\usepackage{graphics}
\usepackage{graphicx}
\usepackage{amssymb}
\usepackage{amsthm}
\usepackage{colortbl}
\usepackage{rotating}
\usepackage{multirow}
\usepackage{multicol}
\usepackage{anysize}
\usepackage{ragged2e}
\usepackage[colorlinks=true,linkcolor=blue,urlcolor=blue,citecolor=blue]{hyperref}
\usepackage{stmaryrd}
\usepackage{setspace}
\usepackage{color}
\usepackage[spanish]{babel}
\usepackage[utf8]{inputenc}
\usepackage[left=0.95in,top=0.95in,right=0.95in,bottom=0.95in]{geometry}

\usepackage{nccmath}

\allowdisplaybreaks
\hypersetup{pdfstartview = FitH} %hace que la hoja se expanda en anchura, y no se muestre toda la hoja al abrirlo

\journal{European Journal of Operational Research}

\begin{document}

\begin{frontmatter}

\newtheorem{lem}{Lema}
\newtheorem{defi}{Definition}
\newtheorem{pro}{Proposition}
\newtheorem{cor}{Corollary}

%\renewcommand{\tablename}{Table}
%\renewcommand{\figurename}{Figure}
%\renewcommand{\abstractname}{Abstract}
%\renewcommand{\refname}{References}

\title{Modelo Determinístico para calendarización de partidos de fútbol}

%\author[label1]{Alejandro Cataldo}
%\author[label2]{Guillermo Durán}
%\author[label3]{Pablo Rey}
%\author[label4]{Tomás Reyes}
%\author[label5]{Antoine Sauré}
%\author[label6]{Denis Sauré}
%\author[label7]{Gustavo Angulo}

%\address[label1]{School of Engineering, Pontificia Universidad Católica de Chile, Av. Vicuña Mackenna 4860, Santiago, Chile. Corresponding Author. Tel: +56 (2) 2354 1234. Email: aecatald@uc.cl}
%\address[label2]{XXX, Universidad de Buenos Aires, XXX, Buenos Aires, Argentina.}
%\address[label3]{XXX, Universidad Tecnológica Metropolitana, XXX, Santiago, Chile.}
%\address[label4]{School of Engineering, Pontificia Universidad Católica de Chile, Av. Vicuña Mackenna 4860, Santiago, Chile.}
%\address[label5]{Telfer School of Management, Universidad de Ottawa, 55 Laurier Ave E, Ottawa, Canadá.}
%\address[label6]{School of Engineering, Universidad de Chile, XXX, Santiago, Chile.}
%\address[label7]{School of Engineering, Pontificia Universidad Católica de Chile, Av. Vicuña Mackenna 4860, Santiago, Chile.}
\end{frontmatter}


\section{Formulation of the Model}\label{FM}
Comenzamos la formulación del modelo definiendo la notación que será utilizada en su formulación.

\vspace{0.05in} Los conjuntos:

\begin{tabular}{p{1.6cm}cp{12.69cm}}
$\mathcal{F}$   & : & conjunto de fechas que restan por jugar en el campeonato.\\
$\mathcal{I}$   & : & conjunto de equipos que conforman el torneo.\\
$\mathcal{N}$   & : & conjunto de partidos (equipo local vs equipo visitante) que aún deben enfrentarse en alguna de las fechas restantes del campeonato.\\
$\mathcal{S}$   & : & conjunto de patrones de localías y visitas posibles en las $K$ fechas que restan por jugar en el torneo.
\end{tabular}

\vspace{0.05in} Los índices:

\begin{tabular}{p{0.79cm}cp{13.5cm}}
$f,l$   & : & índice asociado al conjunto de fechas que restan por jugar $(f \in \mathcal{F})$.\\
$i,j$   & : & índices asociados al conjunto de equipos que conforman el torneo $(i,j \in \mathcal{I})$\\
$n$     & : & índice asociado al conjunto de partidos (equipo local vs equipo visitante) que debe programarse en alguna de las fechas restantes del torneo $(n \in \mathcal{N})$.\\
$s$     & : & índice asociado al conjunto de patrones de localías y visitas $(s \in \mathcal{S})$.
\end{tabular}

\vspace{0.05in} Los parámetros:

\begin{tabular}{p{0.79cm}cp{13.5cm}}
$PI_i$    & : & parámetro discreto que indica la cantidad de puntos que tiene el equipo $i$ justo al terminar la fecha anterior a la primera de las fechas que quedan por jugar.\\
\end{tabular}
\begin{tabular}{p{0.79cm}cp{13.5cm}}
$R_{in}$    & : & parámetro discreto que toma valor 0, 1 o 3, y que corresponde a la cantidad de puntos que gana el equipo $i$ al jugar el partido $n$ (en el que juegan $i$ vs otro equipo).\\
$EL_{in}$   & : & parámetro binario que toma valor 1 si el equipo $i$ es local en el partido $n$, y 0 en cualquier otro caso.\\
$EV_{in}$   & : & parámetro binario que toma valor 1 si el equipo $i$ es visita en el partido $n$, y 0 en cualquier otro caso.\\
$W_{is}$    & : & parámetro binario que toma valor 1 si al equipo $i$ se le puede asignar el patrón de localías y visitas $s$, y 0 en cualquier otro caso.\\
$L_{s}^f$   & : & parámetro binario que toma el valor 1 si el patrón de localías y visitas $s$ indica que el partido es de local en la fecha $f$, y 0 en cualquier otro caso.
\end{tabular}

\vspace{0.05in}

Las variables:

\begin{tabular}{p{0.79cm}cp{13.5cm}}
$x_{n}^f$   & : & variable binaria que toma valor 1 si el partido $n$ se programa en la fecha $f$, y 0 en cualquier otro caso.\\
$y_{is}$    & : & variable binaria que toma valor 1 si al equipo $i$ se le asigna el patrón de localías y visitas $s$, y 0 en cualquier otro caso.\\
$p_{ji}^{lf}$  & : & variable discreta que indica la cantidad de puntos que tiene el equipo $j$ al finalizar la fecha $f$ teniendo información finalizada la fecha $l$ $(f > l)$, en el mejor conjunto de resultados futuros para el equipo $i$.\\
$\hat{p}_{ji}^{lf}$  & : & variable discreta que indica la cantidad de puntos que tiene el equipo $j$ al finalizar la fecha $f$ teniendo información finalizada la fecha $l$ $(f > l)$, en el peor conjunto de resultados futuros para el equipo $i$.\\
$v_{ni}^{lf}$  & : & variable binaria que toma valor 1 si el equipo local gana el partido $n$ de la fecha $f$ teniendo información finalizada la fecha $l$ $(f > l)$, en el mejor conjunto de resultados futuros para el equipo $i$.\\
$a_{ni}^{lf}$  & : & variable binaria que toma valor 1 si el equipo visitante gana el partido $n$ de la fecha $f$ teniendo información finalizada la fecha $l$ $(f > l)$, en el mejor conjunto de resultados futuros para el equipo $i$.\\
$e_{ni}^{lf}$  & : & variable binaria que toma valor 1 si se empata el partido $n$ de la fecha $f$ teniendo información finalizada la fecha $l$ $(f > l)$, en el mejor conjunto de resultados futuros para el equipo $i$.\\
\end{tabular}
\begin{tabular}{p{0.79cm}cp{13.5cm}}
$\hat{v}_{in}^{lf}$  & : & variable binaria que toma valor 1 si el equipo local gana el partido $n$ de la fecha $f$ teniendo información finalizada la fecha $l$ $(f > l)$, en el peor conjunto de resultados futuros para el equipo $i$.\\
$\hat{a}_{in}^{lf}$  & : & variable binaria que toma valor 1 si el equipo visitante gana el partido $n$ de la fecha $f$ teniendo información finalizada la fecha $l$ $(f > l)$, en el peor conjunto de resultados futuros para el equipo $i$.\\
$\hat{e}_{in}^{lf}$  & : & variable binaria que toma valor 1 si se empata el partido $n$ de la fecha $f$ teniendo información finalizada la fecha $l$ $(f > l)$, en el peor conjunto de resultados futuros para el equipo $i$.\\
$\alpha_{ji}^l$  & : & variable binaria que toma valor 1 si el equipo $j$ termina con menos puntos que el equipo $i$ en el mejor conjunto de resultados futuros para el equipo $i$, teniendo información finalizada la fecha $l$.\\
$\hat{\alpha}_{ji}^l$  & : & variable binaria que toma valor 1 si el equipo $j$ termina con menos puntos que el equipo $i$ en el peor conjunto de resultados futuros para el equipo $i$, teniendo información finalizada la fecha $l$.\\
$\beta_{i}^l$  & : & variable discreta que indica la mejor posición que puede alcanzar al final del torneo el equipo $i$ en su mejor conjunto de resultados futuros, teniendo información finalizada la fecha $l$.\\
$\hat{\beta}_{i}^l$  & : & variable discreta que indica la peor posición que puede alcanzar al final del torneo el equipo $i$ en su peor conjunto de resultados futuros, teniendo información finalizada la fecha $l$.\\
\end{tabular}


\newpage
\section{Problema maestro}
\begin{equation}\label{ecFO}
  \mbox{(SSTPA)}\qquad \text{Max } \sum_{l \in \mathcal{F}} \sum_{i \in \mathcal{I}}\left(\hat{\beta}_i^l - \beta_i^l\right)
\end{equation}

~~~~~~~~~~~~s.t:
\begin{equation}\label{ecua1}
    \sum_{f \in \mathcal{F}} x_n^f = 1 \qquad \forall \,n \in \mathcal{N}.
\end{equation}
\begin{equation}\label{ecua2}
    \sum_{n \in \mathcal{N}: EL_{in} + EV_{in}=1} x_n^f = 1 \qquad \forall \,i \in \mathcal{I}; f \in \mathcal{F}.
\end{equation}
\begin{equation}\label{ecua3}
  \sum_{s \in \mathcal{S}: W_{is}=1} y_{is} = 1 \qquad \forall i \in \mathcal{I}.
\end{equation}
\begin{equation}\label{ecua4}
  y_{is} = 0 \qquad \forall i \in \mathcal{I}; s \in \mathcal{S}: W_{i,s} = 0.
\end{equation}
\begin{equation}\label{ecua5}
  \sum_{n \in \mathcal{N}: EL_{in}=1} x_n^f = \sum_{s \in \mathcal{S}: L_s^f = 1} y_{is} \qquad \forall \,i \in \mathcal{I}; f \in \mathcal{F}.
\end{equation}
\begin{equation}\label{ecua6}
  \sum_{n \in \mathcal{N}: EV_{in}=1} x_n^f = \sum_{s \in \mathcal{S}: L_s^f = 0} y_{is} \qquad \forall \,i \in \mathcal{I}; f \in \mathcal{F}.
\end{equation}
%\begin{multline}\label{ecua9}
%  p_{ji}^{lf} = PI_j + \sum_{\theta \in  \mathcal{F}: l \geq \theta}\sum\limits_{n \in \mathcal{N}: EL_{jn}+EV_{jn}=1}R_{jn}x_n^\theta + \sum%\limits_{n \in \mathcal{N}: EL_{jn}=1}\sum\limits_{\theta \in \mathcal{F}: f \geq \theta > l}3v_{ni}^{l\theta} \\
%  + \sum\limits_{n \in \mathcal{N}: EV_{jn}=1}\sum\limits_{\theta \in \mathcal{F}: f \geq \theta > l}3a_{ni}^{l\theta} + \sum\limits_{n \in \mathcal{N}: EL_{jn}+ EV_{jn}=1}\sum\limits_{\theta \in \mathcal{F}: f \geq \theta > l}e_{ni}^{l\theta} \qquad \forall i,j \in \mathcal{I}; f,l \in \mathcal{F}: f > l.
%\end{multline}
%\begin{multline}\label{ecua10}
%  \hat{p}_{ji}^{lf} = PI_j + \sum_{\theta \in  \mathcal{F}: l \geq \theta}\sum\limits_{n \in \mathcal{N}: EL_{jn}+EV_{jn}=1}R_{jn}x_n^\theta + \sum\limits_{n \in \mathcal{N}: EL_{jn}=1}\sum\limits_{l \in \mathcal{F}: f \geq \theta > l}3\hat{v}_{ni}^{l\theta} \\
%  + \sum\limits_{n \in \mathcal{N}: EV_{jn}=1}\sum\limits_{l \in \mathcal{F}: f \geq \theta > l}3\hat{a}_{ni}^{l\theta} + \sum\limits_{n \in \mathcal{N}: EL_{jn}+ EV_{jn}=1}\sum\limits_{l \in \mathcal{F}: f \geq \theta > l}\hat{e}_{ni}^{l\theta} \qquad \forall i,j \in \mathcal{I}; f,l \in \mathcal{F}: f > l.
%\end{multline}
\begin{equation}\label{ecua13}
   \beta_{i}^l = \text{Card$(\mathcal{I})$} - \sum_{j\in \mathcal{I}: j \neq i} \alpha_{ji}^l \qquad \forall i \in \mathcal{I}; l \in \mathcal{F}.
\end{equation}
\begin{equation}\label{ecua14}
   \hat{\beta}_{i}^l = 1 + \sum_{j\in \mathcal{I}: j \neq i} (1-\hat{\alpha}_{ji}^l) \qquad \forall i \in \mathcal{I}; l \in \mathcal{F}.
\end{equation}
\begin{equation}\label{ecua29}
  x_n^f \in\{0,1\} \qquad \forall\, n \in \mathcal{N}; f \in \mathcal{F}.\\
\end{equation}
\begin{equation}\label{ecua30}
  y_{is} \in\{0,1\} \qquad \forall\, i \in \mathcal{I}; s \in \mathcal{S}.\\
\end{equation}
\begin{equation}\label{ecua33}
  \alpha_{ji}^l, \hat{\alpha}_{ji}^l \in \{0,1\} \qquad \forall\, i,j \in \mathcal{I}; l \in \mathcal{F}.\\
\end{equation}
\begin{equation}\label{ecua34}
  \beta_{i}^l, \hat{\beta}_{i}^l \in \mathbb{Z}^+ \qquad \forall\, i \in \mathcal{I}; l \in \mathcal{F}.\\
\end{equation}

Los subproblemas se encargarán de buscar infactibilidades.  Cada subproblema será para todo equipo $i$,  fecha $l$ (se fijan los índices $i$, $l$) en el caso favorable y en el caso desfavorable.  Además de los parámetros mencionado anteriormente,  se añaden los parámetros $\bar{x}_n^f$,$\bar{\alpha}_{ji}^l$ y $\bar{\hat{\alpha}}_{ji}^l$ que son solución del problema maestro.

\section{Subproblema 1}
~~~~~~~~~~~~s.t:
\begin{equation}\label{x_sp1}
x_n^f= \bar{x}_n^f \qquad \forall n \in \mathcal{N}; f \in \mathcal{F}.
\end{equation}
\begin{equation}\label{alpha-sp1}
\alpha_{ji}^l = \bar{\alpha}_{ji}^l \qquad \forall l \in \mathcal{F}; i,j \in \mathcal{I}: i \neq j.
\end{equation}
\begin{equation}\label{ecua7}
 x_n^f = v_{ni}^{lf} + e_{ni}^{lf} + a_{ni}^{lf} \qquad \forall n \in \mathcal{N}; i \in \mathcal{I}; f,l \in \mathcal{F}: f > l.
\end{equation}
\begin{multline}\label{ecua9}
  p_{ji}^{lf} = PI_j + \sum_{\theta \in  \mathcal{F}: l \geq \theta}\sum\limits_{n \in \mathcal{N}: EL_{jn}+EV_{jn}=1}R_{jn}x_n^\theta + \sum\limits_{n \in \mathcal{N}: EL_{jn}=1}\sum\limits_{\theta \in \mathcal{F}: f \geq \theta > l}3v_{ni}^{l\theta} \\
  + \sum\limits_{n \in \mathcal{N}: EV_{jn}=1}\sum\limits_{\theta \in \mathcal{F}: f \geq \theta > l}3a_{ni}^{l\theta} + \sum\limits_{n \in \mathcal{N}: EL_{jn}+ EV_{jn}=1}\sum\limits_{\theta \in \mathcal{F}: f \geq \theta > l}e_{ni}^{l\theta} \qquad \forall i,j \in \mathcal{I}; f,l \in \mathcal{F}: f > l.
\end{multline}
\begin{equation}\label{ecua11}
   M(1 - \alpha_{ji}^l) \geq p_{ii}^{lF} - p_{ji}^{lF} \qquad \forall l \in \mathcal{F}; i,j \in \mathcal{I}: i \neq j.
\end{equation}
\begin{equation}\label{ecua29_sp1}
  x_n^f \in\{0,1\} \qquad \forall\, n \in \mathcal{N}; f \in \mathcal{F}.\\
\end{equation}
\begin{equation}\label{ecua33_sp1}
  \alpha_{ji}^l, \in \{0,1\} \qquad \forall\, i,j \in \mathcal{I}; l \in \mathcal{F}.\\
\end{equation}
\begin{equation}\label{ecua31}
  p_{ji}^{lf} \in \mathbb{Z}^+_0 \qquad \forall\, i,j \in \mathcal{I}; f,l \in \mathcal{F}.\\
\end{equation}
\begin{equation}\label{ecua32}
  v_{ni}^{lf}, e_{ni}^{lf}, a_{ni}^{lf} \in \{0,1\} \qquad \forall\, i \in \mathcal{I}; n \in \mathcal{N}; f,l \in \mathcal{F}.\\
\end{equation}

\section{Subproblema 2}
\begin{equation}\label{x_sp2}
x_n^f= \bar{x}_n^f \qquad \forall n \in \mathcal{N}; f \in \mathcal{F}.
\end{equation}
\begin{equation}\label{alpha-sp2}
\hat{\alpha}_{ji}^l = \bar{\hat{\alpha}}_{ji}^l \qquad \forall l \in \mathcal{F}; i,j \in \mathcal{I}: i \neq j.
\end{equation}
\begin{equation}\label{ecua8}
  x_n^f = \hat{v}_{ni}^{lf} + \hat{e}_{ni}^{lf} + \hat{a}_{ni}^{lf} \qquad \forall n \in \mathcal{N}; i \in \mathcal{I}; f,l \in \mathcal{F}: f > l.
\end{equation}
\begin{multline}\label{ecua10}
  \hat{p}_{ji}^{lf} = PI_j + \sum_{\theta \in  \mathcal{F}: l \geq \theta}\sum\limits_{n \in \mathcal{N}: EL_{jn}+EV_{jn}=1}R_{jn}x_n^\theta + \sum\limits_{n \in \mathcal{N}: EL_{jn}=1}\sum\limits_{l \in \mathcal{F}: f \geq \theta > l}3\hat{v}_{ni}^{l\theta} \\
  + \sum\limits_{n \in \mathcal{N}: EV_{jn}=1}\sum\limits_{l \in \mathcal{F}: f \geq \theta > l}3\hat{a}_{ni}^{l\theta} + \sum\limits_{n \in \mathcal{N}: EL_{jn}+ EV_{jn}=1}\sum\limits_{l \in \mathcal{F}: f \geq \theta > l}\hat{e}_{ni}^{l\theta} \qquad \forall i,j \in \mathcal{I}; f,l \in \mathcal{F}: f > l.
\end{multline}
\begin{equation}\label{ecua12a}
   M \hat{\alpha}_{ji}^l \geq \hat{p}_{ji}^{lF} - \hat{p}_{ii}^{lF} \qquad l \in \mathcal{F}; \forall i,j \in \mathcal{I}: i \neq j.
\end{equation}
\begin{equation}\label{ecua29_sp2}
  x_n^f \in\{0,1\} \qquad \forall\, n \in \mathcal{N}; f \in \mathcal{F}.\\
\end{equation}
\begin{equation}\label{ecua33_sp2}
  \hat{\alpha}_{ji}^l, \in \{0,1\} \qquad \forall\, i,j \in \mathcal{I}; l \in \mathcal{F}.\\
\end{equation}
\begin{equation}\label{ecua31_sp2}
  \hat{p}_{ji}^{f} \in \mathbb{Z}^+_0 \qquad \forall\, i,j \in \mathcal{I}; f,l \in \mathcal{F}.\\
\end{equation}
\begin{equation}\label{ecua32a}
  \hat{v}_{ni}^{lf},  \hat{e}_{ni}^{lf}, \hat{a}_{ni}^{lf} \in \{0,1\} \qquad \forall\, i \in \mathcal{I}; n \in \mathcal{N}; f,l \in \mathcal{F}.\\
\end{equation}
\end{document} 